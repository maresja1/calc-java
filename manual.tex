\documentclass{article}
\usepackage{blindtext}
\usepackage{graphicx}
\usepackage{amsthm}	
\usepackage[utf8]{inputenc} 
\usepackage{float}
\usepackage{listings}


 
\newcommand{\insName}{Bc. Jan Mareš}
\newcommand{\insTitle}{Procedural calculator - User reference}

\title{
\insTitle
}

\author{\insName}
 
\begin{document}
 
\maketitle

\section{Introduction}
The Procedural calculator (hereinafter referred to as calc) is an application allowing
the computation of complex formulas in ten based fractions with adjustable presicision. Calc
is very similar to language interprets with an emphasis put on computation. Classic numeric
formulas are extended with a possibility to define and use functions, variables, conditional
statements or loops.
\section{Basic formulas}
Calc offers computing basic formulas written in an infix form. Supported operators are
+,-,*,/ and brackets. All of them have their usual meaning. See example~\ref{fig:example_basic}.
\begin{figure}[H]
	\texttt{
 		\begin{tabular}{ll}
			Input: & Output: \\
			6+4 & 10 \\
			20/(2+8) & 2 \\
			20/2+8 & 18 \\
			20.3 * 2 & 40.6 \\
		\end{tabular}
	}
	\caption{Basic formula \label{fig:example_basic}}
\end{figure}
\section{Variables}
Basic formulas can be extended with usage of variables. Variables can be assigned a value and
this value can be later retrieved or changed. Reference to variables are not supported. Calc
supports one special variable - $last$. This variable contains value of the last executed 
statement. For usage of variables see exmaple~\ref{fig:example_variables}.
\begin{figure}[H]
	\begin{lstlisting}
a = 20;
b = a + 10;
b;
last;
	\end{lstlisting}
Output: \\
	\texttt{
20 \\
30 \\
30 \\
30 \\
	}
	\caption{Usage of variables \label{fig:example_variables}}
\end{figure}
\section{Functions}
As mentioned in the introduction calc gives to the user ability to define and
call functions. In fuctions variables are by default local - they
are accessible only from this function. Variable can be made global(or parental) with usage of parental
keyword. Function can use named attributes - they behave the same way as local 
variables except their value is set right after calling function. Exmaple~\ref{fig:example_factorial}
shows function factorial written for calc.
\begin{figure}[H]
	\begin{lstlisting}
DEF factorial(n) {
 	f = 1;
 	for(i, 1, n ){
	 	f = f*(i+1)
	}
	f;
}
	\end{lstlisting}
	\caption{Factorial function \label{fig:example_factorial}}
\end{figure}
Code inside a function
can call other functions or itself (also called recursion). Local variables are stored in a stack
per function call. Thanks to this it is safe to use recursion. Example~\ref{fig:example_fibonacci} 
shows usage of recursion when computing Fibonacci's $n$-th number.
\begin{figure}[H]
	\begin{lstlisting}
DEF fibonacci(n){
	if(n<=1){
		1;
	} else {
		fibonacci(n-1)+fibonacci(n-2)
	}
}
	\end{lstlisting}
	\caption{Fibonacci function \label{fig:example_fibonacci}}
\end{figure}
\section{Control flow}
Main program is the code that is not part of any function. In the main program result of
every statement is printed to the output. Every statement must have some result value. Result
value of a function call is the result value of its statement. Result value of a compound statement
is the result value of its last statement.
\section{Syntax}
\subsection{For loop}
For loop is similar to the one used in Pascal. Name
of the iterator variable is specified, then its start value
and upper bound. For loop continues executing its body while 
iterator value is less than the upper bound. Start value or upper bound
can be constant or expression.
\begin{figure}[H]
	\begin{lstlisting}
for(iterator name, start value, upper bound){
	code;
}
	\end{lstlisting}
Example:
	\begin{lstlisting}
for(i,0,5){
	f(i);
}
	\end{lstlisting}
Is equal in C to:
	\begin{lstlisting}
for(int i=0; i<5; ++i){
	f(i);
}
	\end{lstlisting}
	\caption{For loop \label{fig:syntax_for}}
\end{figure}
\subsection{Condition}
Conditions are almost the same as conditions in Java and C.
\begin{figure}[H]
	\begin{lstlisting}
if(expression){
	executed when expression is different from 0
} else {
	executed when expression is 0
}
	\end{lstlisting}
	\caption{For loop \label{fig:syntax_if}}
\end{figure}
\subsection{Grammar}
\begin{figure}[H]
\texttt
{Program $\rightarrow$ Statement StatementList \\
StatementList $\rightarrow$ \textit{;}Statement StatementList | $\lambda$ \\
Statement $\rightarrow$ CompoundStatement | NakedStatement \\
CompoundStatement $\rightarrow$  \{ Statement StatementList \} | \{\} \\
NakedStatement $\rightarrow$ \textit{identifier} = BoolExpression 
| \textit{for} ( \textit{identifier}, BoolExpression, BoolExpression ) Statement \\
| \textit{if}(BoolExpression) Statement IfRest | \textit{parental} \textit{indentifier} | BoolExpression | FuncDef\\
IfRest  $\rightarrow$ \textit{else} Statement | $\lambda$ \\
BoolExpression $\rightarrow$ Expression \textit{==} Expression | Expression \textit{<} Expression | ... for \textit{>,>=,<=} | Exression \\
Expression $\rightarrow$ T \textit{+} T | T \textit{-} T | T \\
T $\rightarrow$ F \textit{*} F | F \textit{/} F | F \\
F $\rightarrow$ (BoolExpression) | \textit{number} | \textit{identifier} | FuncCall \\
FuncDef $\rightarrow$ \textit{DEF} \textit{identifier} (VarList) Statement \\
FuncCall $\rightarrow$ \textit{identifier} (ArgList) \\
VarList $\rightarrow$ \textit{identifer}, VarList | $\lambda$ \\
ArgList $\rightarrow$ BoolExpression, ArgList | $\lambda$ \\}
	\caption{Calc's grammar \label{fig:grammar}}
\end{figure}

\end{document}